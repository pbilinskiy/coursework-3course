\documentclass[main.tex]{subfile}



\begin{document}
	\section{Стаття Bianchi: ергодична збіжність}
	\subsection{Постановка задачі}
	Розглядається сім'я операторів $A(s, x): E \times \mathcal{H} \rightarrow 2^\mathcal{H} $. Наявність першої змінною дає змогу представити випадковий вибір одного оператора із сукупності на кожному кроці.
	
	Уявімо сім'ю операторів, що індексуються множиною $E$: $\{ A_{s} \}_{s \in E}$. У найпростішому випадку $E = {1, 2, \ldots, n}$, тобто, операторів - скінчена кількість. Тепер нехай $(\xi_n(.)) \subset E$ - послідовність випадкових величин, де $\xi_n$ - випадково обраний індекс на кроці $n$. Тоді $A(\xi_n, \cdot)$ - випадково обраний оператор на кроці $n$.
	
	Таким чином, якщо $\xi_n$  реалізує вибір серед $\{1, 2, \ldots, m\}$, де всі наслідки рівноможливі, то це частинний випадок нашої задачі, бо тут обирається лише один оператор із сукупності. 
	
	
	Автор розглядає ітераційну процедуру:
	\begin{equation}
		x_{n+1} = \Res{\lambda_nA(\xi_{n+1}, \cdot)}(x_n)
		\label{bianchi_iter}
	\end{equation}

	
	та прагне довести збіжність майже напевно послідовності середніх за Чезаро:
	
	\[ 		\frac{\sum_{i=1}^{n} \lambda_i x_i}{\sum_{i=1}^{n} \lambda_i}   \]
	
	Збіжність такої послідовності середніх автор називає {\itshape ергодичною збіжністю}.
	
	Автор шукає розв'язки включення 
\begin{equation}
	0 \in \mathbb{E}[ A(\xi, \cdot) ] 
	\label{bianchi_inclusion}
\end{equation} де справа стоїть математичне сподівання Аумана. Для нашого випадку воно має вигляд
	
	\[ \mathbb{E}[ A(\xi, \cdot)] = \frac{1}{m}\sum_{i=1}^{m} A_i(x)\]
	
	Зрозуміло, що тоді включення  \ref{bianchi_inclusion} еквівалентне:
	\[ 0 \in \sum_{i=1}^{m} A_i(x)\]
	
	що є частинним випадком нашої задачі.
	\subsection{Основні поняття}
	\begin{enumerate}
		\item Випадкові величини 
		\item Математичне сподівання Аумана
	\end{enumerate}
	\subsection{Основні результати}
	{\color{red} (in progress)}
	\theoremstyle{corrolary}
	\begin{theorem}
		Нехай виконуються припущення 1-5. Нехай також всі області визначення $D_s$ співпадають для всіх $s$ поза $\mu$-нехтовною (``$\mu$-negligible'') множиною. Нехай $(x_n)$ - послідовність, що дається формулою \ref{bianchi_iter}, а $(\overline{x_n})$ - послідовність середніх Чезаро. .... $ $
	\end{theorem}
    \subsection{Висновок} 
    
    
    Отже, можна використати результати цієї статті для випадку, коли на кожному кроці з сукупності операторів обирається лише один.
\end{document}