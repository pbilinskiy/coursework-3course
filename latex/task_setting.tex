\documentclass[main.tex]{subfile}

\begin{document}
	\section{Постановка задачі}
	Нехай в гільбертовому просторі $\mathcal{H}$ задані багатозначні оператори $A_1, A_2, \ldots, A_m$. Потрібно знайти такі $x \in \mathcal{H}$, що 
	\begin{equation}
		0 \in A_1(x) + A_2(x) + \cdots + A_m(x)
		\label{inclusion_main}
	\end{equation}  
	Зазвичай ми припускаємо, що оператори $A_i$ - монотонні, більше того, максимально монотонні. 
	В роботі \cite{semenov_dec} був запропонований алгоритм:
	
	{\small
		\itshape
		\begin{center}
			 \bfseries Алгоритм
		\end{center}

		
		1) Задаємо $\{\lambda_n\} \subseteq (0, +\infty), \ x_1 \in \mathcal{H}; \ n:=1.$
		
		
		2) Знаходимо елементи
		\[x_{i, n} = J_{\lambda_nA_i}x_n, \ i=\overline{1, m}.\]
		
		3) Покладаємо
		\[x_{n+1} = \frac{1}{m}x_{1, n} + \frac{1}{m}x_{2, n} + \ldots + \frac{1}{m}x_{m, n}\]
		
		$n:=n+1$, переходимо на крок 2.
	}
	
	Також в роботі \cite{semenov_article} було встановлено, що послідовність середніх по Чезаро:
	
	\begin{equation}
		\frac{\sum_{i=1}^{n} \lambda_i x_i}{\sum_{i=1}^{n} \lambda_i}
		\label{chesaro_mean}
	\end{equation}

	
	за певних умов слабко збігається до розв'язку включення \ref{inclusion_main}.
	
	На практиці часто обчислення значень $J_{\lambda_nA_i}x_n$ для всіх $i = \overline{1, m}$ потребує багато обчислювальних ресурсів. Щоб пришвидшити інтераційний процес, на кожному кроці можемо обчислювати лише певну кількість цих значень. Отримаємо 
	{\small
		\itshape
		\begin{center}
			\bfseries Рандомізований алгоритм
		\end{center}
		
		
		1) Задаємо $\{\lambda_n\} \subseteq (0, +\infty), \ x_1 \in \mathcal{H}; \ n:=1.$
		
		2) Випадковим чином вибираємо $k$ індексів $\{i_1, i_2, ..., i_k\}$ з множини $\{1, 2, \ldots, m\}$, $k < m$
		
		3) Для кожного з обраних індексів обчислюємо елементи
		\[x_{i, n} = J_{\lambda_nA_i}x_n, \ i  \in \{i_1, i_2, ..., i_k\}.\]
		
		3) Покладаємо
		\[x_{n+1} = \frac{1}{k}x_{1, n} + \frac{1}{k}x_{2, n} + \ldots + \frac{1}{k}x_{i_k, n}\]
		
		$n:=n+1$, переходимо на крок 2.
	}

	Постає питання про властивості цього алгоритму, а саме:
	\begin{enumerate}
		\item Збіжність ітераційного процесу. Оскільки алгоритм - рандомізований, то доцільніше говорити про збіжність посліовності випадкових величин (в середньому, майже напевно). 
		\item Оцінка швидкості збіжності.
	\end{enumerate}
    
	Потрібно:
	\begin{itemize}
		\item Встановити істотні припущення про оператори $A_i$ та спосіб отримання вибірки.
		\item Сформулювати і довести теореми про збіжність, а також інші проміжні результати.
		\item Розглянути важливі частинні випадки
	\end{itemize}
\end{document}